%%%%%%%%%%%%%%%%%%%%%%%%%%%%%%%%%%%%%%%%%%%%%%%%%%%%%%%%%%%%%%%%%%%%%%%%%%%%%%%%%%
%%																				%%
%% File name: 		introduction.tex											%%
%% Project name:	Hochleistungsantenne										%%
%% Type of work:	T3X00 project work											%%
%% Author:			Sarah Brückner, Maximilian Stiefel, Hannes Bohnengel		%%
%% Date:			07th June 2016												%%
%% University:		DHBW Ravensburg Campus Friedrichshafen						%%
%% Comments:		Created in gedit with tab width = 4							%%
%%																				%%
%%%%%%%%%%%%%%%%%%%%%%%%%%%%%%%%%%%%%%%%%%%%%%%%%%%%%%%%%%%%%%%%%%%%%%%%%%%%%%%%%%

\chapter{Einleitung}
In der Vergangenheit wurden teure Satelliten nur zu Forschungs-- und Verteidigungszwecken entwickelt und gebaut. Heutzutage nimmt der 
Geschäftsanteil mit kommerziellen Satelliten stetig zu. Möchte der Fussball--Fan ein Europameisterschaftsspiel in Echtzeit verfolgen können oder der 
Reisende sich zu seinem Ziel navigieren lassen, so sind Satellitensysteme unverzichtbar.
\newpar
Eine entscheidende Rolle bei der Satellitenforschung haben die 
Amateurfunksatelliten 
gespielt. Neue Techniken konnten ohne ein kommerzielles Geschäft oder eine wissenschaftliche bzw. militärische Mission zu gefährden mit 
Amateurfunksatelliten getestet werden. Die ständige Beobachtung der Satelliten wurde durch die 
Vielzahl der Amateurfunker weltweit sichergestellt. 
\newpar
Um Satelliten in der Nachrichtentechnik nutzen zu können, wird eine Bodenstation benötigt. Mit entsprechender Hard- und 
Software können Satelliten verfolgt werden. Auch eine Kommunikation über und mit Objekten im Orbit ist durch ein entsprechendes Equipment möglich. 
\newpar
Diese Studienarbeit beschreibt die Inbetriebnahme einer freien Software 
zur Satellitenbahnvorhersage und Ansteuerung einer Hochleistungsantenne. Dies schließt die Nachführung der Antenne und damit die Steuerung von 
Rotoren mit ein. Die freie Software soll die bestehende Software 
SatPC32 ersetzen. SatPC32 ist eine proprietäre Software und dient der Bodenstation als Referenz. Da SatPC32 nicht mehr weiterentwickelt wird 
und es darüber hinaus keinen Support für diese Software gibt, wurde der alternative Weg, einer freien Software für die Bodenstation in 
Friedrichshafen, gewählt. Diese freie Software ist GPredict. GPredict ist quelloffen und beherrscht es mit Hilfe der \ac{TLE} die Satellitenposition 
vorherzusagen. Des Weiteren wird von GPerdict bzw. der darunter liegenden HamLib eine Vielzahl verschiedener Hardware-Einheiten zur Ausrichtung der 
Antenne und zur Kommunikation unterstützt. Die Ausgabe der Satellitenposition auf einer grafischen Oberfläche beherrscht GPredict problemlos. Eine 
Kommuniktaion mit Transpondern im Orbit aufzubauen ist mit GPredict kinderleicht.
\newpar
Diese Studienarbeit beschreibt GPredict und soll dem Leser einen 
leichten Einstieg in die Bedienung dieser Software ermöglichen. Außerdem beleuchtet diese Arbeit den physikalischen Hintergrund der 
Satellitenbahn-Vorhersage und damit einhergehend die Ursache der Dopplerverschiebung der Mittenfrequenz bei der Kommunikation mit einem Satelliten. 
Ein weiterer inhaltlicher Grundpfeiler dieser Studienarbeit ist die Dokumentation des gesamten Projektprozess. 
%  In der heutigen Gesellschaft ... Satelliten kaum mehr wegzudenken ... mobile Kommunikation, Navigation, Forschung, Satellitenfernsehen ...\\
%  Bei der Entwicklung und dem Test der Kommunikation zwischen Erde und Satellit haben Amateurfunksatelliten und Amateurfunkbodenstationen eine entscheidende Rolle gespielt (Quelle Amsat-Website)...\\
%  An der Dualen Hochschule Baden-Württemberg am Campus Friedrichshafen soll eine solche Bodenstation errichtet werden.\\
%  Die Software SatPC32 soll ersetzt werden und dient dabei als Referenz.

\clearpage
