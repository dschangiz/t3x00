%%%%%%%%%%%%%%%%%%%%%%%%%%%%%%%%%%%%%%%%%%%%%%%%%%%%%%%%%%%%%%%%%%%%%%%%%%%%%%%%%%
%%																				%%
%% File name: 		conclusion.tex												%%
%% Project name:	Hochleistungsantenne										%%
%% Type of work:	T3X00 project work											%%
%% Author:			Sarah Brückner, Maximilian Stiefel, Hannes Bohnengel		%%    
%% Date:			01st May 2016											    %%
%% University:		DHBW Ravensburg Campus Friedrichshafen						%%
%% Comments:		Created in gedit with tab width = 4							%%
%%																				%%
%%%%%%%%%%%%%%%%%%%%%%%%%%%%%%%%%%%%%%%%%%%%%%%%%%%%%%%%%%%%%%%%%%%%%%%%%%%%%%%%%%

\chapter{Zusammenfassung und Ausblick}
In dieser Studienarbeit, Inbetriebnahme einer freien Software zur Satellitenbahnvorhersage und Ansteuerung einer Hochleistungsantenne, wurde eine 
freie Software validiert und in Betrieb genommen.
Durch eine umfassende Recherche wurde die freie Software GPredict als eine sehr gute Alternative zu 
der Referenzsoftware SatPC32 verifiziert. 
\newpar
GPredict ist ein Satelliten-Tracking Programm und ermöglicht eine Anbindung an die Antennenrotoren sowie 
an das Funkgerät. Unter der Verwendung der \ac{TLE} berechnet GPredict die jeweilige Satellitenbahn. Um zu verstehen, wie GPredict eine 
Bahnvorhersage berechnet, wurden die physikalischen Hintergründe theoretisch beschrieben und die Berechnungen der Keplerelemente hergeleitet.
\newpar
Um GPredict mit der Hardwareumgebung anzubinden, wurde eine entsprechende Konfiguration von GPredict vorgenommen. Die Funkgerät- und Antennensteuerung 
erfolgt mittels der Hamlib--Bibliotheken. Dafür wurden eigene Batch-Skripte geschrieben um die Verwendung der verfügbaren Kommandozeilenprogramme zu 
vereinfachen. 
\newpar
Für die Validierung der Anforderungsdefinition, wurden definierte Tests durchgeführt. Dabei handelte es sich um Modultests, Integrationstests und 
Systemtests. 
\newpar
Zusammenfassend wurde die Studienarbeit gemäß dem V-Modell bearbeitet und strukturiert zu einem Ergebnis gebracht. Dieses Ergebnis korreliert mit der 
Anforderungsdefinition und bietet eine Grundlage für weitere Studienarbeiten an der Bodenstation DHBW Ravensburg Campus Friedrichshafen. Dabei wäre 
eine weitere Aufgabe die Implementierung eines Befehls zum Tausch von Sub-- und Main--Band.
\newpar
Im Rahmen dieser Studienarbeit wurden die Inhalte der Nachrichtentechnik und des technischen Managements reflektiert und ein neues Gebiet der 
Bahnmechanik erschlossen. Für die Unterstützung und Betreuung dieser Studienarbeit möchten wir uns recht herzlich bei Herrn Hardy Lau bedanken.
\clearpage
