%%%%%%%%%%%%%%%%%%%%%%%%%%%%%%%%%%%%%%%%%%%%%%%%%%%%%%%%%%%%%%%%%%%%%%%%%%%%%%%%%%
%%																				%%
%% File name: 		abstract.tex												%%
%% Project name:	Hochleistungsantenne										%%
%% Type of work:	T3X00 project work											%%
%% Author:			Sarah Brückner, Maximilian Stiefel, Hannes Bohnengel		%%
%% Date:			13th July 2016												%%
%% University:		DHBW Ravensburg Campus Friedrichshafen						%%
%% Comments:		Created in gedit with tab width = 4							%%
%%																				%%
%%%%%%%%%%%%%%%%%%%%%%%%%%%%%%%%%%%%%%%%%%%%%%%%%%%%%%%%%%%%%%%%%%%%%%%%%%%%%%%%%%

\chapter*{Kurzfassung}
Diese Studienarbeit umfasst die Inbetriebnahme und Beschreibung einer freien Software zur Satellitenverfolgung und Orbitvorhersage namens GPredict, die Beschreibung der Bodenstation und die Aufarbeitung der theoretischen Hintergründe.\newpar
In Kapitel \ref{chap:projekt} werden Themen des Projektmanagments erläutert, welche bezüglich dieser Studienarbeit eine Rolle spielen. Anschließend wird im Kapitel \ref{chap:einfuehrung} eine Einführung zu den Themen Amateurfunksatelliten und dem Aufbau der Bodenstation der \ac{DHBW} Ravensburg Campus Friedrichshafen dargestellt.\newpar
In Kapitel \ref{hintergruende} wird daraufhin eine umfassende Aufarbeitung der theoretischen Hintergründe zur Bahnmechanik, dem Doppler-Effekt und der Dopplerverschiebung vorgenommen. Ein zentrales Augenmerk wird hierbei auf die Herleitung der Keplerschen Bahnelemente und das Lösen des Kepler-Problems gelegt. Außerdem wird auf die Bestimmung des Vektors zwischen Bodenstation und Satelliten und die von GPredict verwendeten Bahnmodelle näher eingegangen.\newpar
Im nächsten Kapitel wird zunächst die Software GPredict vorgestellt, wobei die Bedienung und die Fähigkeiten genauer erläutert werden. Anschließend wird die Funktion und der Aufbau der Hardware-Bibliothek HamLib beschrieben. Zuletzt erlangt der Leser einen Einblick in die Inbetriebnahme und die damit verbunden Tests und Modifikationen der Software GPredict (und HamLib), bevor im letzten Kapitel eine Zusammenfassung der vorliegenden Arbeit und ein Ausblick für zukünftige Studienarbeiten gegeben wird.